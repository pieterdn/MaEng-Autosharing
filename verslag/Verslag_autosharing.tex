\documentclass[a4paper, 12pt, one column]{article}

%% Language and font encodings. This says how to do hyphenation on end of lines.
\usepackage[dutch]{babel}
\usepackage[utf8x]{inputenc}
\usepackage[T1]{fontenc}
\usepackage{parskip}
% \usepackage[clean]{svg}

%% Sets page size and margins. You can edit this to your liking
\usepackage[top=1.3cm, bottom=2.0cm, outer=2.5cm, inner=2.5cm, heightrounded,
marginparwidth=1.5cm, marginparsep=0.4cm, margin=2.5cm]{geometry}

%% Useful packages
\usepackage[final]{rotating, graphicx} %allows you to use jpg or png images. PDF is still recommended
%\graphicspath{ {./images/} }
\usepackage[colorlinks=False]{hyperref} % add links inside PDF files
\usepackage{amsmath}  % Math fonts
\usepackage{amsfonts} %
\usepackage{amssymb}  %
\usepackage[final]{listings}
\usepackage{xcolor}
\usepackage{authblk}
\usepackage{mathpazo}
%\usepackage[margin=0.25in]{geometry}
\usepackage{pgfplots}
\usepackage{pgfplotstable}
\pgfplotsset{width=10cm,compat=1.9}
\usepackage {tikz}
\usetikzlibrary {positioning}

\definecolor{codegreen}{rgb}{0,0.6,0}
\definecolor{codegray}{rgb}{0.5,0.5,0.5}
\definecolor{codepurple}{rgb}{0.58,0,0.82}
\definecolor{backcolour}{rgb}{0.95,0.95,0.92}
\lstdefinestyle{mystyle}{ 
    commentstyle=\color{codegreen},
    keywordstyle=\color{magenta},
    numberstyle=\tiny\color{codegray},
    stringstyle=\color{codepurple},
    basicstyle=\ttfamily\footnotesize,
    breakatwhitespace=false,         
    breaklines=true,                 
    captionpos=b,                    
    keepspaces=true,                                   
    showspaces=false,                
    showstringspaces=false,
    showtabs=false,                  
    tabsize=1
}

\lstset{style=mystyle}

%% Citation package
\usepackage[authoryear]{natbib}
\bibliographystyle{abbrvnat}
\setcitestyle{authoryear,open={(},close={)}}

\title{Zoekprobleem autosharing: Verslag}
\author{Pieter Dilien, Lucas Van Laer}

\begin{document}
\maketitle

\section{Oplossingsvoorstelling}
De oplossing wordt voorgesteld door een object dat 3 attributen heeft. De eerste is een req\_to\_car array. Hierbij stelt elke index een reservatie voor en elke waarde een wagen. De tweede is een car\_to\_zone array. Hierbij stelt elke index een wagen voor en elke waarde een zone. De laatste is een int die de cost voorstelt.

% Hieronder de oplossingsvoorstelling in Python, aangezien deze analoog is in Rust.
% \begin{lstlisting}[language=Python]
% #Python
% class SolutionModel:
%     req_to_car: npt.NDArray[np.int16]   # Give req as index get car
%     car_to_zone: npt.NDArray[np.int16]  # Give car as index get zone
%     cost: int
% \end{lstlisting}

Deze voorstelling is gebaseerd op de output-file dat gegenereerd moet worden. Daardoor is het wegschrijven naar deze file erg gemakkelijk.

\section{Datavoorstelling}
Om berekeningen te doen werken we met een object dat volgende attributen heeft:
\begin{itemize}
    \item 1D Array of ints: reservation id as index, get corresponding car
    \item 1D Array of ints: car id as index, get corresponding zone
    \item 1D Array of ints: reservation id as index, get corresponding cost
    \item 2D Array of ints: car id as index, get list of corresponding reservations
    \item 2D Array of bools: row for every reservation, every row has length = \#cars
    \item 2D Array of bools: row for every zone, every row has length = \#cars
    \item 1D Array of RequestStructs: reservation id is the index
    \item 1D Array of ZoneStructs: zone id is the index
    \item int: cost
\end{itemize}

Dit in combinatie met twee structs, een RequestStruct en ZoneStruct. De RequestStruct houdt alle informatie bij over 1 reservatie. Dit gaat dan over de zone, dag, starttijd, tijd, lijst van mogelijke wagens, penalty 1 en penalty 2 kost. De ZoneStruct houdt dan weer informatie bij over 1 zone. Hiermee wordt bedoeld: een lijst van aanliggende zones, een booleaanse lijst die de relatie geeft t.o.v. elke andere zone.


\section{Initiele Oplossing}
Om de initiele oplossing te berekenen gaan we sequentieel over de lijst van reservaties. Bij elke reservatie gaan we de verschillende wagens af. Eerst wordt er nagegaan of de wagen reeds aan een zone gelinkt is, indien dit nog niet het geval is, plaatsen we de wagen in de zone van de reservatie. 

Indien de wagen al aan een zone gelinkt is checken we of deze in de (naburige) zone van de reservatie ligt. Indien dit het geval is wordt er gekeken of deze wagen dan ook beschikbaar is, zo ja linken we deze wagen aan de reservatie.

Als er over alle wagens (die mogelijk zijn voor de specifieke reservatie) gegaan is en er geen wagen beschikbaar is, wordt er naar de volgende reservatie gegaan en blijft de reservatie ontoegewezen.

\section{Zoekomgeving}
Onze zoekomgeving werd gedefineerd door twee operatoren. Een kleine die eerder kleine aanpassingen maakt en een grote die grotere sprongen maakt.

\subsection{Kleine Operator}
De kleine operator itereert random over de verschillende reservaties en per reservatie random over de verschillende wagens. Er wordt telkens gecheckt of er geen (beter) wagen kan gevonden worden voor een reservatie. Met beter wordt hier voornamelijk bedoeld een wagen die in de eigen zone van een reservatie ligt en niet in een naburige. De resultaten van enkel deze operator zijn te zien als v1 op \ref{360_res}.

\subsection{Grote Operator}
Deze werd van begin af aan reeds gecombineerd met de kleine operator. Dit door eerst de grote operator uit te voeren en wanneer deze geen verbetering meer gaf de klein verder uit te voeren.

Versie 1: verplaats een random wagen naar een random zone en gaat vervolgens alle reservaties af om te kijken of er een wagen kan aan toegewezen worden. Resultaten te zien als v2 op \ref{360_res}.

Versie 2: iteert random over een lijst met wagens en per wagen random over een lijst met zones. Vervolgens wordt dan hetzelfde gedaan als bij Versie 1. Een groot verschil is dat hierbij de beste wordt gezocht. Alle combinaties worden dus geprobeerd en beste wordt gekozen. Resultaten te zien als v3 op \ref{360_res}. Het is duidelijk te zien op de grafiek dat deze manier slecht werkt op korte tijd, omdat de operator zo veel computing time nodig heeft.

Versie 3: i.p.v. zoals in Versie 2 de beste te zoeken, wordt dit keer de eerste beste genomen. We zien duidelijk op \ref{360_res}, dat v4 die hiermee correspondeerd een beter resultaat geeft.

Versie 4: hierbij werd geen aanpassing gedaan aan de operator zelf, maar aan wanneer deze wordt uitgevoerd in combinatie met de kleine operator. In het begin blijven we de grote operator uitvoeren totdat deze geen verbetering meer geeft. Hierna voeren we de kleine operator 5 keer uit, waarna we de grote nog een keer uitvoeren. Als de grote operator in totaal 2 keer gefaald heeft, slaan we de huidige oplossing op en beginnen we met een nieuwe initiele oplossing. Iets dat we ook pas sinds deze versie doen. We zien op \ref{360_res} dat het resultaat (v5) voornamelijk beter is bij een langere runtime.

\section{Metaheuristiek}
In deze sectie gaan we het kort hebben over de verschillende metaheuristieken die wij geprobeerd hebben.
\subsection{Simulated Annealing (v6)}

Hierbij beginnen we met een start temperatuur van 300 na 1000 iteraties wordt deze gedeeld door 1,3 totdat we een minimum temperatuur 5 of kleiner bereiken of we voor 500 iteraties niet verbeteren.
Nadat de local search stopt herbeginnen we met een andere initiele oplossing. Deze versie bevat ook een simplistische threshold die niet van waarde verandert behalve als we vast komen te zitten,
hetzelfde gebeurt bij de temperatuur. Deze verhoogt als we falen om een betere oplossing te vinden.

\subsection{Thresholding (v7)}

In deze versie hebben we alleen maar voor thresholding gekozen, hierdoor ging de cost sneller naar beneden en hebben we onze beste cost gevonden.
Deze versie houdt ook geen rekening met of het vast komt te lopen of niet. We zagen dat als we dit er wel bij staken op dezelfde manier als bij versie 6 dat dit eigenlijk niks deed.
Het verschil met deze thresholding en met de simplistische thresholding die we in versie 6 deden is dat deze wel daalt. De threshold daalt met 2 na 2000 iteraties en start op 40.
Als de threshold 20 of lager gaat proberen we opnieuw met een andere initiele oplossing.

\section{Resultaten}
Hieronder kan u de resultaten zien voor de finale versie (v7), dit is de versie met thresholding. De resultaten zijn voor 360\_5\_71\_25.csv met telkens een runtime van 5 minuten en een random seed.

\begin{center}
\pgfplotstabletypeset{final_data.dat}
\end{center}
\section{Bijlagen}

% \begin{figure}[ht]
% \begin{center}
% \begin{tikzpicture}
% \begin{axis}[
%     width=3.5in,
%     ylabel={Cost}, 
%     xlabel={Runtime [min]},
%     ytick={10000,11000,12000,13000},
%     ymajorgrids=true,
%     grid style=dashed,
%     tick label style={/pgf/number format/fixed},
%     scaled ticks=false,
%     legend pos=outer north east,
%     ]
%     \addplot table [y=v1, x=Time]{data_100.dat};
%     \addlegendentry{Version 1}
%     \addplot table [y=v2, x=Time]{data_100.dat};
%     \addlegendentry{Version 2}
%     \addplot table [y=v3, x=Time]{data_100.dat};
%     \addlegendentry{Version 3}
%     \addplot table [y=v4, x=Time]{data_100.dat};
%     \addlegendentry{Version 4}
%     \addplot table [y=v5, x=Time]{data_100.dat};
%     \addlegendentry{Version 5}
%     \addplot table [y=v6, x=Time]{data_100.dat};
%     \addlegendentry{Version 6}
%     \addplot table [y=v7, x=Time]{data_100.dat};
%     \addlegendentry{Version 7}
% \end{axis}
% \end{tikzpicture}
% \end{center}
% \caption{Results for 100\_5\_14\_25.csv with seed 123}
% \end{figure}

\begin{figure}[ht]
\begin{center}
\begin{tikzpicture}
    \begin{axis}[
        width=3.5in,
        ylabel={Cost}, 
        xlabel={Runtime [min]},
        ytick={10000,15000, 20000, 25000},
        ymajorgrids=true,
        grid style=dashed,
        tick label style={/pgf/number format/fixed},
        scaled ticks=false,
        legend pos=outer north east,
        ]
        \addplot table [y=v1, x=Time]{data_360.dat};
        \addlegendentry{v1 (Python)}
        \addplot table [y=v2, x=Time]{data_360.dat};
        \addlegendentry{v2 (Python)}
        \addplot table [y=v3, x=Time]{data_360.dat};
        \addlegendentry{v3 (Python)}
        \addplot table [y=v4, x=Time]{data_360.dat};
        \addlegendentry{v4 (Python)}
        \addplot table [y=v5, x=Time]{data_360.dat};
        \addlegendentry{v5 (Python)}
        \addplot table [y=v6, x=Time]{data_360.dat};
        \addlegendentry{v6 (Rust)}
        \addplot table [y=v7, x=Time]{data_100.dat};
        \addlegendentry{v7 (Rust)}
    \end{axis}
\end{tikzpicture}
\caption{Results for 360\_5\_71\_25.csv with seed 123}
\label{360_res}
\end{center}
\end{figure}

% \begin{filecontents*}{100_seeds.csv}
% seed, cost
% \end{filecontents*}

% \begin{tikzpicture}
%     \begin{axis}[
%         ymin = 9500, ymax = 11000,
%         xmin = 0, xmax = 100
%     ]
%     \addplot table [x=seed, y=cost, col sep=comma] {100_seeds.csv};
%     \end{axis}
% \end{tikzpicture}

\end{document}
